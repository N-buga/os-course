\begin{frame}[fragile]
\frametitle{Атомарные операции}

\begin{itemize}
  \item Compare-And-Set атомарные инструкции:
    \begin{lstlisting}
int cmpxchg(volatile int *ptr, int old, int new)
{
  const int stored = *ptr;
  if (stored == old)
    *ptr = new;
  return stored;
}
    \end{lstlisting}
  \item Пара инструкций Load Linked и Store Conditional:
    \begin{lstlisting}
int cpu = INVALID_CPU, *track = 0;

int load_linked(volatile int *ptr)
{
  cpu = this_cpu();
  return *(track = ptr);
}

int store_conditional(volatile int *ptr, int value)
{
  if (track != ptr || cpu != this_cpu())
    return 0;
  cpu = INVALID_CPU;
  track = 0;
  *ptr = value;
  return 1;
}
    \end{lstlisting}
\end{itemize}
\end{frame}

\begin{frame}
\frametitle{Атомарные операции}

\begin{itemize}
  \item Примеры Compare-And-Set инструкций:
        \begin{itemize}
          \item \emph{cmpxchg} - архитектура x86 (используйте префикс "lock",
                чтобы инструкция служила барьером памяти);
          \item \emph{cas} - архитектура sparc (является барьером);
        \end{itemize}
  \item примеры Load Linked и Store Conditional:
        \begin{itemize}
          \item \emph{ldrex}/\emph{strex} - архитектура ARM (не является
                барьером - требуют явного барьера перед и после, если нужно);
          \item \emph{lwarx}/\emph{stwcx} - архитектура PowerPC (не являются
                барьером - требуют явного барьреа перед и после, если нужно);
        \end{itemize}
\end{itemize}
\end{frame}

\begin{frame}[fragile]
\frametitle{Атомарные операции}

Compare-And-Set операцию можно выразить через пару Load Linked/Store
Conditional:
\begin{lstlisting}
int cmpxchg(volatile int *ptr, int old, int new)
{
  do {
    const int stored = load_linked(ptr);

    if (stored != old)
      return stored;
  } while (!store_conditional(ptr, new));
  return old;
}
\end{lstlisting}
\end{frame}

\begin{frame}
\frametitle{Другие атомарные операции}

Обычно процессоры вместе с Load Linked/Store Conditional и Compare-And-Set могут
предоставлять дургие атомарные операции:

\begin{itemize}
  \item xchg - атомарный обмен значениями, в переменную записывается новое
        значение, а возвращается старое;
  \item add/sub/inc/dec - атомарные арифметические операции;
\end{itemize}

\onslide<2>{Все эти операции можно реализовать используя Compare-And-Set или Load Linked/Store Conditional.}
\end{frame}

\begin{frame}[fragile]
\frametitle{Дргие атомарные операции}

Реализация xchg через cmpxchg:
\begin{lstlisting}
int xchg(volatile int *ptr, int new_value)
{
    int old_value;

    do {
        old_value = *ptr;
    } while (cmpxchg(ptr, old_value, new_value) != old_value);

    return old_value;
}
\end{lstlisting}

Реализация xchg через load\_linked и store\_conditional:
\begin{lstlisting}
int xchg(volatile int *ptr, int new_value)
{
    int old_value;

    do {
        old_value = load_linked(ptr);
    } while (!store_conditional(ptr, new_value));

    return old_value;
}
\end{lstlisting}
\end{frame}
