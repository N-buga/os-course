\begin{frame}
\frametitle{Неблокирующая синхронизация}

Зачем нам неблокирующая синхронизация?
\begin{itemize}
  \item deadlock-и иногда случаются
    \begin{itemize}
      \item возможно ошибка в алгоритме (такое случается, но нам не особо интересно);
      \item поток исполнения может упасть в критической секции (хотя мы и опирались явно на обратное);
    \end{itemize}
  \item инверсия приоритетов:
    \begin{itemize}
      \item менее приоритетный поток держит Lock, и не может ее отпустить, потому что ему не дают процессорное время;
    \end{itemize}
\end{itemize}
\end{frame}

\begin{frame}
\frametitle{Неблокирующая синхронизация}

Можно ли обходиться без Lock-ов?
\begin{itemize}
  \item<1> да, конечно, у нас ведь есть cmpxchg - нужно просто вызывать его в правильном порядке;
  \item<2> все, конечно, не так просто:
    \begin{itemize}
      \item правильный порядок не всегда очевиден;
      \item не каждый порядок можно считать неблокирующим - нам нужно более формальное определение;
      \item придется решать новые проблемы, которых не было при использовании Lock-ов;
      \item результат может быть непрактичным (медленным).
    \end{itemize}
\end{itemize}
\end{frame}

\begin{frame}
\frametitle{Формальные определения}

\only<1>{\begin{definition}
Реализация алгоритма/структуры данных obstruction free, если начиная с любого достижимого состояния, любой поток исполнения может завершить
выполнение задачи за конечное количество шагов, если \emph{все другие потоки остановятся}.
\end{definition}}

\only<2>{\begin{definition}
Реализация lock free, если \emph{один из потоков} (не известно какой) гарантированно завершает выполнение своей задачи за конечное число шагов,
\emph{независимо от других потоков}.
\end{definition}}

\only<3>{\begin{definition}
Реализация wait free, если \emph{каждый поток} завершает свою операцию за конечное количество шагов \emph{независимо от других потоков}.
\end{definition}}
\end{frame}

\begin{frame}[fragile]
\frametitle{Lock free stack}
\lstinputlisting[language=C++,firstline=8,lastline=25]{../lockless/stack1.hpp}
\end{frame}

\begin{frame}[fragile]
\frametitle{Lock free stack}
\lstinputlisting[language=C++,firstline=39,lastline=49]{../lockless/stack1.hpp}
\end{frame}

\begin{frame}[fragile]
\frametitle{Lock free stack}
\lstinputlisting[language=C++,firstline=51,lastline=67]{../lockless/stack1.hpp}
\end{frame}

\begin{frame}
\frametitle{Lock free stack}

К сожалению эта реализация:
\begin{itemize}
  \item<1-> не obstruction/lock/wait free (возможно);
    \begin{itemize}
      \item<2-> чтобы реализация была lock free, она не должна использовать не lock free примитивов как минимум;
      \item<3-> является ли вызов new lock free?
    \end{itemize}
  \item<4-> не корректна:
    \begin{itemize}
      \item<5-> обращается к памяти после освобождения;
      \item<6-> неправильно использует Compare-And-Set примитив для синхронизации;
    \end{itemize}
\end{itemize}
\end{frame}

\begin{frame}[fragile]
\frametitle{Безопасное освобождение памяти}

\begin{lstlisting}
node = this->head.load(std::memory_order_relaxed);
if (!node)
    return nullptr;
next = node->next;
\end{lstlisting}
Имеем ли мы право разыменовывать указатель node?
\begin{itemize}
  \item несколько вызовов pop могут выполняться параллельно;
  \item один из них мог освободить память, после того как мы прочитали node и до того, как мы прочитали node->next;
\end{itemize}
\end{frame}

\begin{frame}[fragile]
\frametitle{Проблема ABA}
\begin{lstlisting}
node = this->head.load(std::memory_order_relaxed);
if (!node)
    return nullptr;
next = node->next;
\end{lstlisting}
Гораздо более тонкая неприятность:
\begin{itemize}
  \item<1-> compare\_exchange\_strong успешен, если значение хранимое в атомарной переменной \emph{побитово совпадает} с ожидаемым значением (это упрощение);
  \item<2-> нам нужно, чтобы compare\_exchange\_strong был успешен, если значение в атомарной переменной \emph{не менялось} с момента, когда мы сделали load (как в LL/SC);
\end{itemize}
\end{frame}

\begin{frame}
\frametitle{Проблема ABA}

\begin{itemize}
  \item У нас есть несколько потоков работающих со стеком параллельно:
    \begin{itemize}
      \item Thread0 выполняют операцию pop;
      \item Thread1 выполняет операцию pop и затем сразу операцию push;
      \item все остальные потоки каким-то образом модифицируют стек;
    \end{itemize}
\end{itemize}
\end{frame}

\begin{frame}[fragile]
\frametitle{Проблема ABA}

\begin{lstlisting}
node = this->head.load(std::memory_order_relaxed);
if (!node)
    return nullptr;
next = node->next;
\end{lstlisting}

\begin{itemize}
  \item Thread0 загрузил голову стека (пусть ее адрес Addr, т. е. node == Addr);
  \item Thread0 проверил, что она не равна nullptr, т. е. стек не пуст;
  \item Thread0 запоминает указатель на следующий элемент стека;
  \item после чего Thread0 был снят с процессора;
\end{itemize}
\end{frame}

\begin{frame}
\frametitle{Проблема ABA}

\begin{itemize}
  \item Thread1 выполняет операцию pop полностью:
    \begin{itemize}
      \item удаляет голову стека из списка;
      \item освобождает память занятую головой списка, т. е. Addr теперь свободный адрес;
    \end{itemize}
  \item Thread1 снимается с процессора и управление получают другие потоки, которые изменяют стек;
  \item Thread1 снова получает управление и выполняет операцию push полностью:
    \begin{itemize}
      \item Thread1 должен алоцировать новую голову стека, в частности Addr - свободный адрес;
      \item Thread1 добавляет новую голову с адресом Addr в стек;
    \end{itemize}
\end{itemize}
\end{frame}

\begin{frame}[fragile]
\frametitle{Проблема ABA}

\begin{lstlisting}
do {
    ...
} while (!this->head.compare_exchange_strong(node, next));
\end{lstlisting}

\begin{itemize}
  \item Thread0 вновь получает процессор и продолжает исполнение операции pop;
  \item Thread0 выполняет compare\_exchage\_strong - операция успешна, потому что после Thread1 адрес головы стека вновь стал Addr;
  \item<2-> но поле next в голове стека могло измениться!
\end{itemize}
\end{frame}
