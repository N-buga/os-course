\begin{frame}
\frametitle{Разделяемая память}

\begin{itemize}
  \item<1-> Рассмотренные варианты IPC обладают рядом недостатков:
    \begin{itemize}
      \item они либо не позволяют передавать данные вообще либо требуют копирования;
      \item они требуют взаимодействия с ядром (системного вызова) на передачу и на получение данных
        \begin{itemize}
          \item системные вызовы довольно дешевы;
          \item но плохо, если приходится делать их слишком часто;
        \end{itemize}
    \end{itemize}
  \item<2-> Адресные пространтсва процессов по умолчанию отделены друг от друга
    \begin{itemize}
      \item но мы можем попросить ОС создать общий участок памяти;
      \item общая память самый дешевый способ взаимодействия.
    \end{itemize}
\end{itemize}
\end{frame}

\begin{frame}
\frametitle{Разделяемая память}

\begin{itemize}
  \item<1-> Мы можем попросить ОС выделить именованный участко памяти
    \begin{itemize}
      \item по имени, этот участок памяти смогут найти другие процессы;
      \item для создания участка разделенной памяти используется shm\_open;
      \item для удаления используется shm\_unlink - вся память хранится до удаления участка или перзагрузки;
    \end{itemize}
  \item<2-> процессы могут просить ОС отобразить этот участок памяти в адресное пространство процесса
    \begin{itemize}
      \item чтобы отобразить регион в адресное пространство используется mmap;
      \item чтобы убрать отображение используется munmap;
      \item с помощью mmap можно также отображать файлы в память - очень быстрый способ чтения/записи;
    \end{itemize}
\end{itemize}
\end{frame}

\begin{frame}[fragile]
\frametitle{Разделяемая память}
\lstinputlisting[language=C++,firstline=13,lastline=28]{../ipc/shmem.c}
\end{frame}
