\begin{frame}
\frametitle{Каналы (a. k. a. Pipes)}

\begin{itemize}
  \item<1-> Сигналы и коды возврата не позволяют передавать данные
    \begin{itemize}
      \item в отсуствие общей памяти, это может стать проблемой;
    \end{itemize}
  \item<2-> Чтобы передавать данные между процессами есть множество способов:
    \begin{itemize}
      \item каналы или pipes должны быть вам уже хорошо известны;
      \item вы можете создать участок общей памяти;
      \item вы можете использовать сокеты;
      \item вы можете использовать файлы (хотя это очень не удобно);
    \end{itemize}
\end{itemize}
\end{frame}

\begin{frame}
\frametitle{Pipes}

\begin{itemize}
  \item Pipe - это пара файловых дескрипторов:
    \begin{itemize}
      \item в один дескриптор можно писать поток байт;
      \item из другого млжно читать (очевидно, то что было записано в первый);
      \item pipe имеет ограниченный буффер;
      \item pipe не сохраняет границы сообщений (по-умолчанию);
    \end{itemize}
\end{itemize}
\end{frame}

\begin{frame}[fragile]
\frametitle{Pipes}
\lstinputlisting[language=C++]{../ipc/pipe.c}
\end{frame}

\begin{frame}
\frametitle{Named pipes}

\begin{itemize}
  \item<1-> Pipe можно сделать перманентным:
    \begin{itemize}
      \item для этого существует вызов mkfifo;
      \item кроме того есть команда mkfifo (можно вызвать из bash);
    \end{itemize}
  \item<2-> Именованный канал выглядит почти как обычный файл:
    \begin{itemize}
      \item в отличие от файла, чтение из канала удаляет данные;
      \item смещение в канале не имеет смысла (seek не работает);
      \item именованный канал не хранит данные на диске;
    \end{itemize}
\end{itemize}
\end{frame}
