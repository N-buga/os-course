\begin{frame}
\frametitle{Сигналы}

\begin{itemize}
  \item<1-> Сигналы - сообщения о каких-либо исключительных событиях (как правило):
    \begin{itemize}
      \item SIGSEGV - segmentation fault (не очень оно исклчительное);
      \item SIGCHLD - статус дочернего процесса изменился;
      \item SIGTERM и SIGKILL - процессу пора на покой;
    \end{itemize}
  \item<2-> обработчики сигналов асинхронны по отношению к основному коду:
    \begin{itemize}
      \item в этом смысле они очень походи на прерывания;
      \item в отличие от прерываний их можно сделать синхронными;
    \end{itemize}
  \item<3-> не все сигналы можно обработать или проигнорировать:
    \begin{itemize}
      \item SIGTERM можно перехватить, а SIGKILL нельзя.
    \end{itemize}
\end{itemize}
\end{frame}

\begin{frame}[fragile]
\frametitle{Пример обработчика сигналов}
\lstinputlisting[language=C++]{../ipc/alive.c}
\end{frame}

\begin{frame}
\frametitle{Сигналы и конкурентность}

\begin{itemize}
  \item<1-> Как и прерывания, асинхронные сигналы являются источником конкуренции:
    \begin{itemize}
      \item обработчик сигнала может прервать исполнение кода;
      \item код может быть не готов к тому, что его прервут;
    \end{itemize}
  \item<2-> На обработчики сигналов налагаются ограничения:
    \begin{itemize}
      \item например, в обработчике сигнала нельзя вызывать функцию printf;
      \item многие функции стандартной библиотеки могут быть не "signal-safe";
      \item создавать "signal-safe" функции возможно.
    \end{itemize}
\end{itemize}
\end{frame}

\begin{frame}[fragile]
\frametitle{Маскировка сигналов}
\lstinputlisting[language=C++]{../ipc/block.c}
\end{frame}

\begin{frame}
\frametitle{Синхронная обработка сигналов}

\begin{itemize}
  \item<1-> Можно погрузить поток в сон в ожидании сигнала:
    \begin{itemize}
      \item sigwait или sigwaitinfo позволяют дождаться нужных сигналов;
      \item в многопоточных программах, не редко, создают поток для обработки сигналов:
        \begin{itemize}
          \item при старте процесса все сигналы блокируются;
          \item выделеный поток с помощью sigwait/sigwaitinfo ожидает сигналов и нотифицирует другие потоки, если нужно;
        \end{itemize}
    \end{itemize}
  \item<2-> В Linux есть специальный системный вызов signalfd:
    \begin{itemize}
      \item создает фиктивный файловый дескриптор, из которого можно "читать" сигналы;
    \end{itemize}
\end{itemize}
\end{frame}

\begin{frame}
\frametitle{Сигналы и IPC}

Процессы могут посылать сигналы дургим процессам:
\begin{itemize}
  \item SIGKILL, SIGTERM - чтобы "попросить" их умереть;
  \item SIGUSR1 и SIGUSR2 - специальные сигналы, выделенные под нужды приложений.
\end{itemize}
\end{frame}

\begin{frame}[fragile]
\frametitle{Отправка сигналов}
\lstinputlisting[language=C++]{../ipc/kill.c}
\end{frame}
