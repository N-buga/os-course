\begin{frame}
\frametitle{Сокеты}

\begin{itemize}
  \item<1-> Pipe-ы не позволяют отличать "отправителей":
    \begin{itemize}
      \item вы не знаете, кто записал байт в pipe, вы видите только байт;
    \end{itemize}
  \item<2-> Pipe-ы не сохраняют границы сообщений:
    \begin{itemize}
      \item у вас есть поток байт, но вам не известно как он был создан;
      \item в Linux можно создать pipe, который будет сохранять границы;
    \end{itemize}
\end{itemize}
\end{frame}

\begin{frame}
\frametitle{Сокеты}

\begin{itemize}
  \item<1-> Сокеты предоставляют более широкий выбор возможностей по сравнению с pipe-ами:
    \begin{itemize}
      \item вы можете выбирать сохранять или нет границы сообщений (DGRAM vs STREAM);
      \item вы можете узанать, от кого пришли данные;
      \item вы можете передавать данные между компьютерами;
    \end{itemize}
  \item<2-> В частности, сокеты предоставляют доступ к сетевым сервисам ОС:
    \begin{itemize}
      \item сокеты это интрефейс, за ним не обязательно должна быть настоящая сеть;
      \item но зачастую сокеты ассоциируются именно с сетевым взаимодействием.
    \end{itemize}
\end{itemize}
\end{frame}

\begin{frame}
\frametitle{Сокеты}

\begin{itemize}
  \item<1-> Для создания сокета используется вызов socket
    \begin{itemize}
      \item при создании нужно указать "домен" или семейство протоколов;
      \item нужно указать тип соединения (DGRAM или STREAM);
      \item можно укзать конкретный проткол, который будет использован при передаче;
    \end{itemize}
  \item<2->Примеры семейств протоколов:
    \begin{itemize}
      \item AF\_UNIX, AF\_LOCAL, AF\_NETLINK - скорее всего вам не знакомы;
      \item AF\_INET, AF\_INET6 - известны вам под именами IPv4 (TCP/IP) и IPv6;
      \item AF\_BLUETOOTH - может быть и такое.
    \end{itemize}
\end{itemize}
\end{frame}

\begin{frame}
\frametitle{Тип соединения}

\begin{itemize}
  \item<1-> При создании сокета вы указываете тип соединения:
    \begin{itemize}
      \item тип соединения позволяет выбрать нужный протокол передачи из семейства протоколов;
      \item если с семейством протоколов AF\_INET вы укажите тип SOCK\_STREAM, то будет использован протокол TCP;
      \item если укажите SOCK\_DGRAM, то будет использован UDP.
    \end{itemize}
  \item<2-> В общем случае тип соединения описывает гарантии:
    \begin{itemize}
      \item гарантии сохранения границ сообщений;
      \item гарантии доставки - можем отправить и забыть, а можем ждать подтверждения доставки и переотправлять сообщения по таймаутам;
    \end{itemize}
\end{itemize}
\end{frame}

\begin{frame}
\frametitle{Адресация}
\begin{itemize}
  \item<1-> Чтобы отправлять и получать сообщения нам нужны адреса:
    \begin{itemize}
      \item чтобы отправить сообщение именно на тот сокет, который нужно;
      \item чтобы кто-то мог отправить сообщение на наш сокет;
    \end{itemize}
  \item<2-> Адрес, ествественно, зависит от семейства протоколов:
    \begin{itemize}
      \item вы не можете полагаться на то, что bluetooth использует тот же формат адресов что и IP;
      \item в частности для TCP/IP адрес состоит из IP адреса и порта;
    \end{itemize}
\end{itemize}
\end{frame}

\begin{frame}
\frametitle{Адресация}
\begin{itemize}
  \item<1-> Для указания адресов есть два вызова:
    \begin{itemize}
      \item bind задает "наш" адрес - адрес, по которму могут связаться с нашим процессом;
      \item connect задачет "чужой" адрес - адрес процесса, с которым мы хотим связаться;
      \item сокет, таким образом, описывает соединение и определяется парой адресов и протоколом;
    \end{itemize}
  \item<2-> bind и connect нужны не всегда:
    \begin{itemize}
      \item если вы иницируете соединение, то bind может быть не обязателен, а такой сокет часто называют клиентским;
      \item если вы создаете сокет, чтобы кто-то мог подключиться к вам, то connect не нужен, а такой сокет часто называют серверным;
    \end{itemize}
\end{itemize}
\end{frame}

\begin{frame}
\frametitle{Пример: серверный сокет}

\begin{itemize}
  \item<1-> Создадим сокет использующий протокол TCP для соединения;
    \begin{itemize}
      \item в качестве порта будем использовать 2016;
      \item в качестве IP адреса возьмем 127.0.0.1;
      \item не каждый порт можно использовать на свое усмотрение;
    \end{itemize}
  \item<2-> серверный сокет используется для приема входящих соединений:
    \begin{itemize}
      \item чтобы указать, что сокет используется для приема входящих соединений используется вызов listen;
      \item чтобы дождаться входящего соединения используется вызов accept;
      \item accept возвращает новый сокет, который описывает установленное соединение;
    \end{itemize}
\end{itemize}
\end{frame}

\begin{frame}[fragile]
\frametitle{Пример: серверный сокет}
\lstinputlisting[language=C++,firstline=22,lastline=43]{../ipc/serv.c}
\end{frame}

\begin{frame}[fragile]
\frametitle{Пример: серверный сокет}
\lstinputlisting[language=C++,firstline=45,lastline=62]{../ipc/serv.c}
\end{frame}

\begin{frame}
\frametitle{Пример: клиентский сокет}

\begin{itemize}
  \item У нас есть сервер, пусть наш клиент подключается к нему:
    \begin{itemize}
      \item т. е. в качестве удаленного адреса будем использовать 127.0.0.1:2016
      \item локальный адрес указывать не обязательно;
    \end{itemize}
  \item передавать будем параметры командной строки;
\end{itemize}
\end{frame}

\begin{frame}[fragile]
\frametitle{Пример: клиентский сокет}
\lstinputlisting[language=C++,firstline=12,lastline=33]{../ipc/clnt.c}
\end{frame}
