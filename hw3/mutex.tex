\section{Примитив взаимного исключения}

Примитив взаимного исключения потоков должен предоставлять как минимум функции
lock и unlock, которыми можно оградить использование разделяемых ресурсов. Кроме
очевидных гарантий взаимного исключения, требуется, чтобы вызовы lock/unlock
могли быть вложенными, т. е. такое использование допустимо:

\begin{verbatim}
lock(lock_descriptor0)
    lock(lock_descriptor1)

    unlock(lock_descriptor1)
unlock(lock_descriptor0)
\end{verbatim}

Поддерживать следующий вариант можно, но совершенно не обязательно:

\begin{verbatim}
lock(lock_descriptor0)
    lock(lock_descriptor1)

unlock(lock_descriptor0)
    unlock(lock_descriptor1)
\end{verbatim}

Параметры\footnote{Да и имена функции lock/unlock.} остаются на ваше усмотрение.

При реализации lock и unlock, настоятельно рекомендуется\footnote{Это
очень-очень-очень упростит вам жизнь.} помнить, что вы работаете с
однопроцессорной системой, т. е. настоящего параллельного исполнения нет.
