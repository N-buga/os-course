\begin{frame}
\frametitle{Page Cache}
\begin{itemize}
  \item<1-> Пользователь, обычно, не обращается к диску - он обращается к файлу
    \begin{itemize}
      \item где и как этот файл располагается на диске не особо важно;
      \item ФС использует Buffer Cache для ускорения работы с диском;
    \end{itemize}
  \item<2-> Зачем просить ФС делать лишнюю работу и искать нужный сектор диска?
    \begin{itemize}
      \item давайте кешировать доступ к файлам вместо доступа к диску;
      \item в Linux Kernel это называется Page Cache (судя по всему название
            пришло из SVR4);
    \end{itemize}
\end{itemize}
\end{frame}

\begin{frame}[fragile]
\frametitle{Page Cache в Linux Kernel}
\begin{lstlisting}
struct address_space {
  struct inode                          *host;
  struct radix_tree_root                page_tree;

  spinlock_t                            tree_lock;
  atomic_t                              i_mmap_writable;
  struct rb_root                        i_mmap;
  struct rw_semaphore                   i_mmap_rwsem;
  unsigned long                         nrpages;
  unsigned long                         nrexceptional;
  pgoff_t                               writeback_index;

  const struct address_space_operations *a_ops;

  unsigned long                         flags;
  spinlock_t                            private_lock;
  struct list_head                      private_list;
  void                                  *private_data;
};
\end{lstlisting}
\end{frame}

\begin{frame}[fragile]
\frametitle{Page Cache в Linux Kernel}
\begin{lstlisting}
struct address_space_operations {
  int (*writepage)(struct page *page, struct writeback_control *wbc);
  int (*readpage)(struct file *, struct page *);
  int (*writepages)(struct address_space *, struct writeback_control *);

  int (*set_page_dirty)(struct page *page);

  int (*readpages)(struct file *filp, struct address_space *mapping,
                   struct list_head *pages, unsigned nr_pages);

  int (*write_begin)(struct file *, struct address_space *mapping,
                     loff_t pos, unsigned len, unsigned flags,
                     struct page **pagep, void **fsdata);
  int (*write_end)(struct file *, struct address_space *mapping,
                   loff_t pos, unsigned len, unsigned copied,
                   struct page *page, void *fsdata);

  ...
};
\end{lstlisting}
\end{frame}

\begin{frame}
\frametitle{Откуда берутся address\_space\_operations?}
\begin{itemize}
  \item Их выставляет ФС при открытии файла:
    \begin{itemize}
      \item ядро алоцирует структуру \emph{struct inode};
      \item \emph{struct address\_space} - поле этой структуры;
      \item ФС заполняет \emph{struct inode} при открытии файла;
   \end{itemize}
\end{itemize}
\end{frame}

\begin{frame}[fragile]
\frametitle{Чтение файла}
\begin{lstlisting}
static ssize_t do_generic_file_read(struct file *filp, loff_t *ppos,
                                    struct iov_iter *iter, ...)
{
  ...
  index = *ppos >> PAGE_CACHE_SHIFT;
  ...
  page = find_get_page(mapping, index);
  ...
  error = mapping->a_ops->readpage(filp, page);
  ...
  ret = copy_page_to_iter(page, offset, nr, iter);
  ...
}
\end{lstlisting}
\end{frame}

\begin{frame}
\frametitle{Запись файла}
\begin{itemize}
  \item В отличие от чтения, запись в файл более сложный процесс
    \begin{itemize}
      \item запись состоит из двух частей: запись в кеш и сброс кеша на диск;
    \end{itemize}
  \item \emph{write\_begin}/\emph{write\_end} и \emph{set\_page\_dirty}
        используются в первой части - при записи в кеш;
  \item \emph{writepage}/\emph{writepages} используются для записи страниц кеша
        на диск;
\end{itemize}
\end{frame}

\begin{frame}[fragile]
\frametitle{Запись файла}
\begin{lstlisting}
ssize_t generic_perform_write(struct file *file,
                              struct iov_iter *i, loff_t pos)
{
  ...
  status = a_ops->write_begin(file, mapping, pos, bytes, flags,
                              &page, &fsdata);
  ...
  copied = iov_iter_copy_from_user_atomic(page, i, offset, bytes);
  ...
  status = a_ops->write_end(file, mapping, pos, bytes, copied,
                            page, fsdata);
  ...
}
\end{lstlisting}
\begin{itemize}
  \item \emph{write\_begin} занимается подготовкой к записи:
    \begin{itemize}
      \item ищет/аллоцирует страницы в Page Cache;
      \item читает данные - запрос на запись может быть не выровнен;
    \end{itemize}
  \item \emph{write\_end} освобождает ресурсы и обновляет метаданные;
\end{itemize}
\end{frame}

\begin{frame}
\frametitle{Запись файла}
\begin{itemize}
  \item<1-> Оставшаяся часть - запись страниц файла на диск:
    \begin{itemize}
      \item никакого секрета или хитрого трюка;
      \item Linux Kernel честно поддерживает список "грязных" inode;
      \item для каждого inode Page Cache отслеживает "грязные"
            страницы;
      \item для записи используется \emph{writepage} из
            \emph{address\_space\_operations};
    \end{itemize}
  \item<2-> Запись страниц происходит:
    \begin{itemize}
      \item по требованию - sync;
      \item по необходимости - мало памяти или много "грязных" страниц;
      \item периодически - мы не хотим, чтобы данные долго оставались
            "грязными" в кеше;
    \end{itemize}
\end{itemize}
\end{frame}

\begin{frame}
\frametitle{Последствия кеширования}
\begin{itemize}
  \item<1-> Важно закрывать файлы:
    \begin{itemize}
      \item это позволит сбросить буфер библиотеки, которую вы
            используете;
      \item это позволит освободить файловый дескриптор - их количество
            ограничено;
      \item это не гарантирует, что файлы записались на диск;
    \end{itemize}
  \item<2-> Как гарантировать, что все закешированные данные \emph{файла}
            реально записались на диск?
    \begin{itemize}
      \item для этого есть два вызова \emph{fsync}/\emph{fdatasync};
      \item если вы только создали файл, то даже это не поможет;
      \item вам нужно также сделать \emph{fsync} на родительском каталоге;
    \end{itemize}
\end{itemize}
\end{frame}
